%%%%%%%%%%%%%%%%%%%%%%%%%%%%%%%%%%%%%%%%%%%%%%%%%%%%%%%%%%%%%%%%%%%%%%%%%%%%%%%
%     STYLE POUR LES EXPOSES TECHNIQUES 
%         3e année INSA de Rennes
%
%             NE PAS MODIFIER
%%%%%%%%%%%%%%%%%%%%%%%%%%%%%%%%%%%%%%%%%%%%%%%%%%%%%%%%%%%%%%%%%%%%%%%%%%%%%%%

\documentclass[a4paper,11pt]{article}

\usepackage{exptech}       % Fichier (./exptech.sty) contenant les styles pour 
                           % l'expose technique (ne pas le modifier)

%\linespread{1,6}          % Pour une version destinée à un relecteur,
                           % décommenter cette commande (double interligne) 
                           
% UTILISEZ SPELL (correcteur orthographique) à accès simplifié depuis XEmacs

%%%%%%%%%%%%%%%%%%%%%%%%%%%%%%%%%%%%%%%%%%%%%%%%%%%%%%%%%%%%%%%%%%%%%%%%%%%%%%%

\title{ \textbf{Réalisation d'un jeu vidéo d'escalade afin de valoriser un travail de recherche en 
animation procédurale} }
\markright{Style pour l'exposé technique} 

%\author{François \textsc{Boschet}, Thomas \textsc{Giraudeau}, \\
%        Delphine \textsc{Millet}, Paul \textsc{Rivière} \\
%        Kevin \textsc{Thek} \\
%        \\
%        Encadrant : Steve \textsc{Tonneau}}

\date{}                    % Ne pas modifier
 
%%%%%%%%%%%%%%%%%%%%%%%%%%%%%%%%%%%%%%%%%%%%%%%%%%%%%%%%%%%%%%%%%%%%%%%%%%%%%%%

\begin{document}          

\maketitle                 % Génère le titre
\thispagestyle{empty}      % Supprime le numéro de page sur la 1re page

\section{Mouvements dans le mode normal}
Dans le mode normal, nous utilisons des animations créées avec le logiciel Blender. Vu que le sol est plat, le mouvement reste réaliste. Les animations concernées sont : l’attente, la marche et le saut. 

\section{Cinématique inverse}
L’algorithme de cinématique inverse CCD a été implémenté dans le jeu sous la forme du script CCD3d.cs . 

\section{Passage d’un mode à l’autre}
Le choix de passage d’un mode à l’autre est dicté par la pression d’une touche par le joueur. À ce moment-là, le script <INSERER> s’active, et procède au redressement du chat, de telle sorte qu’il soit dans une posture humaine. 

\section{Choix des prises}
Nous utilisons la méthode de notre encadrant, implémenté ici sous le nom de CatManipulatibility.cs
<insérer algo >

\section{Intégration des scènes}
Notre mode de développement consistait en la création d’une scène par partie de parcours. Ainsi le mur d’escalade ou encore le cube ont été fait chacun dans une scène différente. Nous avons dû exporter chaque scène comme prefab. Chaque prefab contient les informations de la scène, comme la disposition des éléments, les textures, etc.
Les prefabs ont été assemblées dans la scène finale pour donner lieu au niveau global.

\section{Menu et Interface Utilisateur}

\bibliography{biblio}


\end{document}

%%%%%%%%%%%%%%%%%%%%%%%%%%%%%%%%%%%%%%%%%%%%%%%%%%%%%%%%%%%%%%%%%%%%%%%%%%%%%%%
