%%%%%%%%%%%%%%%%%%%%%%%%%%%%%%%%%%%%%%%%%%%%%%%%%%%%%%%%%%%%%%%%%%%%%%%%%%%%%%%
%     STYLE POUR LES EXPOSES TECHNIQUES 
%         3e année INSA de Rennes
%
%             NE PAS MODIFIER
%%%%%%%%%%%%%%%%%%%%%%%%%%%%%%%%%%%%%%%%%%%%%%%%%%%%%%%%%%%%%%%%%%%%%%%%%%%%%%%

\documentclass[a4paper,11pt]{article}

\usepackage{exptech}       % Fichier (./exptech.sty) contenant les styles pour 
                           % l'expose technique (ne pas le modifier)

%\linespread{1,6}          % Pour une version destinée à un relecteur,
                           % décommenter cette commande (double interligne) 
                           
% UTILISEZ SPELL (correcteur orthographique) à accès simplifié depuis XEmacs
\usepackage{graphicx}

%%%%%%%%%%%%%%%%%%%%%%%%%%%%%%%%%%%%%%%%%%%%%%%%%%%%%%%%%%%%%%%%%%%%%%%%%%%%%%%

\title{ \textsc{Kat'dventures} -- Documentation technique }
\markright{Etude pratique 2014} 

%\author{François \textsc{Boschet}, Thomas \textsc{Giraudeau}, \\
%        Delphine \textsc{Millet}, Paul \textsc{Rivière} \\
%        Kevin \textsc{Thek} \\
%        \\
%        Encadrant : Steve \textsc{Tonneau}}

\date{}                    % Ne pas modifier
 
%%%%%%%%%%%%%%%%%%%%%%%%%%%%%%%%%%%%%%%%%%%%%%%%%%%%%%%%%%%%%%%%%%%%%%%%%%%%%%%

\begin{document}          

\maketitle                 % Génère le titre
\thispagestyle{empty}      % Supprime le numéro de page sur la 1re page

\section{Mouvements dans le mode normal}
Dans le mode normal, nous utilisons des animations créées avec le logiciel Blender. Vu que le sol est plat, le mouvement reste réaliste. Les animations concernées sont : l’attente, la marche et le saut. 

\section{Cinématique inverse}
L'algorithme de cinématique inverse CCD a été implémenté dans le jeu sous la forme du script \texttt{CCD3d.cs}.
\paragraph{Description de l'algorithme CCD} 
\subparagraph{Nomenclature}
\begin{itemize}
\item $p_i$ est la position d'une articulation $q_i$ du bras ;
\item $p_e$ est la position de l'effecteur (c'est-à-dire du bout du bras) ;
\item $p_t$ est la position de la cible à atteindre par l'effecteur. 
\end{itemize} 

\subparagraph{Déroulement de l'algorithme}


\begin{flushleft}
Point de départ : l'articulation la plus proche de l'effecteur ($q_{n-1}$).
\end{flushleft}
\begin{itemize}
\item On calcule $\alpha$ l'angle formé entre les deux vecteurs $\vec{q_i p_e}$ et $\vec{q_i p_t}$
\item On fait une rotation de centre $p_i$ et d'angle $\alpha$ autour de l'axe $\vec{q_i p_e}\land\vec{q_i p_t}$, ce qui permet d'aligner $\vec{q_i p_e}$ et $\vec{q_i p_t}$
\item Si la distance entre l'effecteur et la cible (c'est-à-dire $||\vec{p_e p_t}||$) devient plus petite par rapport à un nombre $\epsilon$ fixé, l'objectif est atteint. 
\item Dans le cas contraire, deux choix : 
\begin{itemize}
\item Si on n'est pas arrivé à la racine du membre ($i>0$), on itère une autre fois avec l'articulation $q_{i-1}$.
\item Si $i<0$, cela veut dire qu'on est arrivé à la racine du membre. On recommence alors avec $q_{n-1}$.
\end{itemize}
\end{itemize}

La figure \ref{ccd} montre un exemple d'application de l'algorithme.

\begin{figure}[htbp]
	$$
	\includegraphics[scale=0.6]{Images/CCDALL.eps}
	$$
	\caption{Application de l'algorithme CCD}
	\label{ccd}
\end{figure}


\section{Passage d'un mode à l’autre}
Le choix de passage d'un mode à l'autre est dicté par la pression d'une touche par le joueur. À ce moment-là, le script \texttt{EnablePush.cs} s'active, et procède au redressement du chat, de telle sorte qu'il soit dans une posture humaine. 

\section{Mouvements en mode super-pouvoirs}
Pour faire bouger le chat en position debout, nous nous sommes appuyés sur la bibliothèque \textit{Locomotion System} développé par Rune Skovbo Johansen\footnote{Le package \textit{Locomotion System} est disponible sur le Unity Asset Store : \textit{https://www.assetstore.unity3d.com/en/\#!/content/7135}}. Cet ensemble de scripts utilise la cinématique inverse pour obtenir un mouvement réaliste et similaire à celui d'un humain.

\section{Choix des prises}
Nous utilisons la méthode de notre encadrant, implémenté ici sous le nom de \texttt{CatManipulatibility.cs}. Nous avons utilisé ce script comme une boîte noire, car nous l'avons reçu tard dans le projet, et notre encadrant a considéré que son implémentation serait trop prenante. Il est à noter que la méthode complète permet d'obtenir une configuration complète (c'est-à-dire la cible et les positions/rotations des articulations), mais nous nous en servons que pour trouver la cible adéquate. 

La fonction qui nous intéresse est \texttt{ftr}, qui prend en paramètre un \texttt{Transform} et la direction que nous voulons donner au chat, et renvoie un indicateur de "réalisme" de la configuration. Cela nous permet de trouver la meilleure prise. Ensuite, les déplacements des articulations sont effectués par les opérations de \texttt{CCD3d.cs}. 

L'algorithme utilisé est le suivant : 
\begin{enumerate}
\item On choisit les prises à proximité (écart d'un $\epsilon$ défini)
\item Pour chaque prise $p$ : 
\begin{enumerate}
\item On crée un clone du membre, et on lui fait décrire l'algorithme CCD pour qu'il trouve une configuration
\item On appelle la fonction \texttt{ftr} du script \texttt{CatManipulability} avec la direction voulue de l'utilisateur. 
\item Si la valeur renvoyée par \texttt{ftr} est plus grande que pour la prise d'avant, on retient la prise comme étant la meilleure.
\end{enumerate}
\end{enumerate}


\section{Intégration des scènes}
Notre mode de développement consistait en la création d’une scène par partie de parcours. Ainsi le mur d’escalade ou encore le cube ont été fait chacun dans une scène différente. 

\section{Menu et Interface Utilisateur}

Les menus sont composés d'une image de fond et de boutons. La position des boutons est fixé "à la main", relativement à la résolution de l'écran. La taille des boutons est géré par un script. Lorsque l'utilisateur passe la souris au dessus des boutons la taille de ceux-ci augmente d'un facteur 1.2. Lorsque l'utilisateur clique sur un bouton, il déclenche un événement qui fait changer de scène.

\end{document}

%%%%%%%%%%%%%%%%%%%%%%%%%%%%%%%%%%%%%%%%%%%%%%%%%%%%%%%%%%%%%%%%%%%%%%%%%%%%%%%
