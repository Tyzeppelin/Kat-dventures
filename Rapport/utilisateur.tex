%%%%%%%%%%%%%%%%%%%%%%%%%%%%%%%%%%%%%%%%%%%%%%%%%%%%%%%%%%%%%%%%%%%%%%%%%%%%%%%
%     STYLE POUR LES EXPOSES TECHNIQUES 
%         3e année INSA de Rennes
%
%             NE PAS MODIFIER
%%%%%%%%%%%%%%%%%%%%%%%%%%%%%%%%%%%%%%%%%%%%%%%%%%%%%%%%%%%%%%%%%%%%%%%%%%%%%%%

\documentclass[a4paper,11pt]{article}

\usepackage{exptech}       % Fichier (./exptech.sty) contenant les styles pour 
                           % l'expose technique (ne pas le modifier)

%\linespread{1,6}          % Pour une version destinée à un relecteur,
                           % décommenter cette commande (double interligne) 
                           
% UTILISEZ SPELL (correcteur orthographique) à accès simplifié depuis XEmacs

%%%%%%%%%%%%%%%%%%%%%%%%%%%%%%%%%%%%%%%%%%%%%%%%%%%%%%%%%%%%%%%%%%%%%%%%%%%%%%%

\title{ \textsc{Kat'dventures} -- Documentation Utilisateur}
\markright{Etude pratique 2014} 

%\author{François \textsc{Boschet}, Thomas \textsc{Giraudeau}, \\
%        Delphine \textsc{Millet}, Paul \textsc{Rivière} \\
%        Kevin \textsc{Thek} \\
%        \\
%        Encadrant : Steve \textsc{Tonneau}}

\date{}                    % Ne pas modifier
 
%%%%%%%%%%%%%%%%%%%%%%%%%%%%%%%%%%%%%%%%%%%%%%%%%%%%%%%%%%%%%%%%%%%%%%%%%%%%%%%

\begin{document}          

\maketitle                 % Génère le titre
\thispagestyle{empty}      % Supprime le numéro de page sur la 1re page

%\begin{center}
%\textsc{Documentation utilisateur}
%\end{center}

Kat’dventures est un jeu vidéo de type plate-forme, c’est-à-dire que l’objectif est l’accomplissement de plusieurs niveaux de jeu dans lesquels l'accent est mis sur l'habileté du joueur à contrôler le déplacement du personnage principal. Dans Kat’dventures, le héros est un chat et son objectif est d’atteindre la pelote de laine dont il suit un fil à travers les différents niveaux du jeu.
Cette documentation présente les différentes possibilités de notre jeu vidéo.

\section{Lancement du jeu}
\paragraph{Lancer le jeu} 	 Lorsqu'on arrive sur le menu principal, on a le choix entre plusieurs boutons :

\begin{itemize}
\item Le bouton Jouer qui permet d'acceder au jeu
\item Le bouton Crédits qui ammène à la scène des crédits
\item Le bouton Quitter qui permet de quitter le jeu
\end{itemize} 

\paragraph{Mettre en pause}  un second menu [resume, quit, …]
\paragraph{Choisir un niveau}  les niveaux vont s’enchainer je pense
\paragraph{Quitter le jeu} 

\section{Commandes du jeu}
\subsection{Mode ordinaire}
\paragraph{Avancer }: \fbox{$\rightarrow$}

\paragraph{Sauter} : \fbox{Espace}

\paragraph{Se mettre debout} : \fbox{Insérer touche ici}
\subsection{Mode super pouvoirs}
\paragraph{Avancer} : \fbox{$\rightarrow$}

\paragraph{Escalader} : Lorsque vous vous approchez suffisament du mur, le chat s'agrippe à une prise. Pour vous déplacer sur le mur, vous pouvez utiliser les touches fléchées : \fbox{$\uparrow$} \fbox{$\rightarrow$} \fbox{$\downarrow$} \fbox{$\leftarrow$}

\paragraph{Pousser} : INSERER DEMARCHE 

\paragraph{Sauter} : \fbox{Espace}



\section{Astuces}
\subsection{Niveau 1}
\begin{enumerate}
\item Passer en mode super pouvoirs puis pousser la caisse dans le trou
\item Sauter par dessus le trou
\item Escalader le mur à l'aide des poignées
\end{enumerate}


\end{document}

%%%%%%%%%%%%%%%%%%%%%%%%%%%%%%%%%%%%%%%%%%%%%%%%%%%%%%%%%%%%%%%%%%%%%%%%%%%%%%%
